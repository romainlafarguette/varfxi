% Time-stamp: <2020-12-03 11:03:42 RLafarguette>
% LateX template 
% August 2020, Romain Lafarguette, rlafarguette@imf.org

%% ---------------------------------------------------------------------------
%% Preamble: Packages and Setup
%% ---------------------------------------------------------------------------
% Document class and font
\documentclass[11pt]{article}

% Language
\usepackage[english]{babel}

% Layout
%\usepackage[margin=1in]{geometry}
\usepackage[DIV=12]{typearea} % Page layout & margin set-up
\usepackage{setspace} % Linespread
\onehalfspacing % x 1,5 spacing

% Font and encoding
\usepackage[utf8]{inputenc} % For input
\usepackage[T1]{fontenc} % For outut
\usepackage{lmodern} % Standard LateX font
\usepackage[babel=true]{microtype}  % "Subliminal refinements towards typographical perfection"
\microtypesetup{final, tracking=true, kerning=true, spacing=true}

% Graphics
\usepackage{graphicx} % Insert graphics
\usepackage{subfig} % Multiple graphs in one figure
\graphicspath{{./figures/logo/}{./figures/output/}} % Relative paths for graphics

% Tables
\usepackage{booktabs, rotating, multirow, pdflscape} % Tabular rules and other macros

% Floats, captions and footnotes
\usepackage{float, caption, capt-of} 

% Put the charts at the end
% \usepackage[nomarkers]{endfloat}

% Mathematics 
\usepackage{amsmath, amsthm, amssymb, mathtools, bbm} % American Mathematical Society macros 

% Subfiles and separate compilations
%\usepackage{subfiles}

% Manage authors and affiliations
\usepackage{authblk} % Properly align institutions name in the title page

% References and hyper references 
\usepackage{url} % Insert url
\usepackage{natbib} % Year-name format
\usepackage{color}  % Color package
\definecolor{darkblue}{rgb}{0,0,0.44} % Customized color for external hyper references
\definecolor{burgundy}{rgb}{0.5, 0.0, 0.13} % Customized color for internal hyper references

% Hyperlinks in pdf
\usepackage[pdftex]{thumbpdf} % Thumbnails
\usepackage[%
    bookmarks=true,   % bookmarks
    bookmarksnumbered=false,
    pdfpagemode=, % bookmark closed at the opening
    pdfstartview=FitH, % all the heigth
    pdfpagelayout=SinglePage, % perpage view
    colorlinks=true, % Coloured links
    breaklinks=true, % new line for too long links
    urlcolor= darkblue, % external links color
    citecolor=darkblue, % bibliography citations
    linkcolor=burgundy, % Internal links color
    pdfborder={0 0 0}   % No borders
    ]{hyperref} % Extensive support for hypertext in LaTeX

\hypersetup{% PDF metadata
  pdftitle = {Foreign Exchange Intervention Rule for Central Banks: A Risk-Based Framework},
  pdfauthor = {Lafarguette and Veyrune},
  pdfkeywords = {Foreign Exchange Intervention Rule for Central Banks: A Risk-Based Framework},
  pdfsubject = {Finance, Financial Economics},
  pdfcreator = {Lafarguette and Veyrune},
  pdfproducer = {Lafarguette and Veyrune},
  pdfpagemode = {UseOutlines}, % When opening pdf. UseOutlines, FullScreen, etc. 
  pdfstartview = {Fit}}

% Footnote without marker, for the title
\newcommand\blfootnote[1]{%
  \begingroup
  \renewcommand\thefootnote{}\footnote{#1}%
  \addtocounter{footnote}{-1}%
  \endgroup
}

% A few macros: environments
\newenvironment{wideitemize}{\itemize\addtolength{\itemsep}{10pt}}{\enditemize}
\newenvironment{wideenumerate}{\enumerate\addtolength{\itemsep}{10pt}}{\endenumerate}

%% ---------------------------------------------------------------------------
%% Title, authors and affiliations
%% ---------------------------------------------------------------------------
\title{\textbf{Foreign Exchange Intervention Rule for Central Banks:
A Risk-Based Framework}}

\author{Romain Lafarguette}
\author{Romain Veyrune}
\affil{International Monetary Fund} 
\date{December 2020}

%% ---------------------------------------------------------------------------
%% Begin the document
%% ---------------------------------------------------------------------------
\begin{document}

\maketitle      \blfootnote{Contacts:      \url{rlafarguette@imf.org}      and
\url{rveyrune@imf.org}. Both of the Monetary and Capital Markets Department of
the IMF.  The authors thank  Suman Basu, Dimitris Drakopoulos,  Kelly Eckhold,
Chris Erceg, Andres Gonzales, Simon  Gray, Darryl King, Vladimir Klyuev, Jorge
Kriljenko,  Istvan Mak,  Thomas McGregor,  Stephen Mulema,  Umang Rawat,  Olga
Stankova,  and  Kevin Wiseman  for  their  comments  and insights.  Karen  Lee
provided research  assistance. The public  data and Python codes  to replicate
the      results      of      this       paper      are      available      at
https://romainlafarguette.github.io/software/.  The  views  expressed  in  IMF
Working Papers are those of the authors and do not necessarily represent the
views of the IMF, its Executive Board, or IMF management.}

%% ---------------------------------------------------------------------------
%% Abstract
%% ---------------------------------------------------------------------------

\begin{abstract} This  paper presents  a new rule  for central  banks’ foreign
exchange (FX) interventions, using the concept of Value at Risk (VaR). The VaR
rule is  used to design an  intervention policy that consistently  transfers a
given share of exchange rate risk from  the market to the central bank balance
sheet, depending  on the economy exposure  to exchange rate risk.  A VaR-based
intervention  rule is  desirable for  countries under  floating exchange  rate
arrangements,  where central  banks intervene  in  the FX  market to  preserve
financial stability. This approach is  consistent with the price and financial
stability mandates of  many central banks, including  inflation targeters. The
VaR  rule has  other appealing  features  for central  banks, including  being
forward  looking and  budget neutral  over the  medium term.  The VaR  rule is
back-tested on Banco Mexico’s publicly available FX interventions data between
2008  and  2016,  both  with  and  without  a  preannounced  fixed  volatility
threshold.\\
\end{abstract}
\bigskip

\noindent \textbf{Keywords:} Foreign Exchange Interventions, Value at Risk, Exchange Rate at Risk, GARCH 

\medskip

\noindent \textbf{JEL classification:} E58 (Central Banks and Their Policies), F31 (Foreign Exchange), G17 (Financial Forecasting and Simulations)

%% ---------------------------------------------------------------------------
%% Suppress page number on title page 
%% ---------------------------------------------------------------------------
\thispagestyle{empty} 
\newpage
\pagenumbering{arabic}

%% ---------------------------------------------------------------------------
%% Introduction
%% ---------------------------------------------------------------------------
\section{Introduction}
\label{sec:introduction}

The 2018 IMF  Annual Report on Exchange Arrangement  and Exchange Restrictions
classified 65  exchange arrangements as  either floating or free  floating, of
which 38 implemented  inflation targeting.  In those  arrangements, the supply
and demand  in the foreign  exchange (FX)  market determine the  exchange rate
with no predictable path.  As a result,  the exchange rate risk is not managed
by the  authorities and  remains fully  with the  private sector,  despite the
financial stability risk  that it entails. As hinted by  the breakdown between
floating  and “free”  floating, not  all  central banks  are comfortable  with
ruling out participation in the FX market. Even those with a strong commitment
to floating, for  example, Chile, Mexico, and Norway, did  intervene in the FX
market   in   cases   of   exceptional    stress,   such   as   the   COVID-19
pandemic.\footnote{See,                      for                      example,
https://www.fx-markets.com/foreign-exchange/4624581/chile-launches-biggest-fx-intervention-in-20-years,
https://www.wsj.com/articles/norwegian-krone-soars-amid-signs-of-central-bank-intervention-11585068173}\\

Typical FX intervention objectives in a floating exchange rate arrangement are
variants of  preserving market  functioning, for example,  smoothing excessive
exchange  rate  volatility and  addressing  disorderly  market conditions,  or
correcting exchange rate misalignment. The  literature adds objectives such as
an adequate stock  of reserves, external competitiveness,  and price stability
(\cite{patel2019},  \cite{chamon2019}). These  all circle  back to  notions of
market  stability  (having enough  foreign  reserves  to intervene  later)  or
exchange rate  equilibrium (external and  price stability), both of  which are
underpinned by the notion of macrofinancial risk to the economy.\\

The main  contribution of this  paper is to  look at intervention  through the
prism of risk. Each economy presents  to a different extent unhedged exposures
to  exchange rate  risk. Unhedged  exposures  include any  direct or  indirect
exposition to  exchange rate risk  by any economic agents.  They fundamentally
depend on structural features of the  economies, chiefly their size and degree
of openness. The degree of resilience  to foreign exchange risk―that is, large
swings in the  exchange rate―that the economy can  absorb varies substantially
across countries.  As interventions  in the FX  spot market  transfer exchange
rate  risk off  the  market, FX  interventions  can be  desirable  even for  a
floating exchange  rate regime,  as maintaining orderly  conditions on  the FX
market is part of a broad financial stability mandate.\\

Central banks usually  keep a fair degree of opacity  about their intervention
triggers, which  makes it difficult to  determine how much exchange  rate risk
they   consider  the   market   could  manage   on  its   own   and  when   to
intervene. However,  it could be  inferred from actual intervention  if enough
data on  interventions are published.  Other  central banks, such as  those in
Colombia,   Guatemala,  and   Mexico,  use   transparent  intervention   rules
\citep{chamon2019} that reveal, at least partially, their risk tolerance.\\

This  paper proposes  an  empirical  methodology based  on  Value  at Risk  to
determine  the  trigger of  an  FX  intervention  rule  anchored to  the  risk
tolerance of the central bank (the VaR rule) that absorbs a constant amount of
exchange rate risk and leaves the rest in the market. This risk-based strategy
is beneficial both from a financial stability angle and because of its support
of the development of the FX risk hedging market. This methodology can also be
used  to  reverse  engineer  central  banks' risk  tolerance  based  on  their
intervention in the FX spot market.\\

There are important conceptual caveats to  mention. Our paper does not explore
the efficiency of FX interventions, as  we use a simple reduced-form framework
without causal identification.  The paper  focuses on the timing (the trigger)
for intervention  and not  on the optimal  intervention amount.   Further work
could explore these issues.\\

This paper relates to the  rule-versus-discretion literature. The paper argues
that  the VaR  rule offers  many of  the benefits  attached to  the rule-based
approach  while  minimizing common  pitfalls  of  rules.   In this  sense,  it
contributes to  constant efforts  of central  banks to  improve the  design of
policy rules  \citep{taylor2017} in the  domain of exchange rate  policy.  The
most remarkable advantage  of the VaR-based intervention is that  it keeps the
risk  transfers constant,  while it  would increase  or decrease,  possibility
without   limit,  with   exchange  rate   volatility  under   fixed-volatility
triggers.\\

The  empirical methodology  is applied  to the  case of  Mexico. The  peso has
floated for a long period in one  of the most liquid FX markets among emerging
economies.  In  addition,  Banco   Mexico's  (BM)  website  provides  detailed
intervention data since 2008. Banco Mexico also implemented interventions both
with a  minimum bid rate (that  is, a preannounced fixed  volatility rule) and
without  a  minimum  bid  rate,   providing  a  diversity  of  experiences  in
intervention  strategy.  Our  paper  focuses  on  only  one  dimension  of  FX
interventions: preventing  excessive volatility  on the FX  market benchmarked
against a risk-based metric. Other relevant  reasons and motives may well have
been factored into BM intervention without a minimum price.\\

The rest  of the paper  is organized as follows:  Section \ref{sec:literature}
presents  a  brief  literature  review.   Section  \ref{sec:var-interventions}
explains the concept of the exchange rate at risk and the formalization of the
FX intervention rule.  Section IV presents the empirical framework  based on a
GARCH model.  Section  V provides the operational framework  for central banks
using the model for their FX interventions. Section VI back-tests the model on
Mexico public data. Section VII concludes.\\

%% ---------------------------------------------------------------------------
%% Literature Review
%% ---------------------------------------------------------------------------
\section{Literature Review}
\label{sec:literature}

The literature on policy rules mainly focuses on the monetary policy decision.
The  debate is  essentially whether  the  monetary policy  decision should  be
guided by a  pre-established reaction function (the rule)  or by policymakers’
expert judgment (discretion).  The reasoning is that rules reduce  the cost of
disinflation  policy  if the  monetary  authorities  have not  established  an
inflation aversion reputation by curbing  an inflation expectation of rational
economic agents \citep{kydland1977}.  Once  a reputation has been established,
the  rule  may  not  be  superior   to  discretion  based  on  sound  judgment
\citep{barro1983}.\\

While less  explored in the literature,  the same arguments could  apply to FX
intervention  rules. They  can  likely  signal the  commitment  to a  floating
exchange rate (within boundaries) and convince rational economic agents (which
may challenge  the central bank  commitment to  float) that the  exchange rate
will experience at least a certain degree of volatility. In addition, FX rules
serve to anchor market expectations \citep{montoro2013} and  to provide
some sense of safety to the market, thereby contributing to its stability.\\

More  generally,  central banks’  commitment  can  be  used to  steer  agents’
behavior. For example, \cite{krugman1991} shows that a commitment to intervene
as the  exchange rate leaves  a target zone causes  change in the  behavior of
economic  agents,  even  when  there   is  no  explicit  intervention.   Also,
\cite{fanelli2020} show  that commitment to future  interventions is necessary
to have an impact on exchange rates today.  Finally, \cite{basu2018} show that
commitment  has additional  benefits over  discretion when  there are  capital
outflows and FX reserves may run out.\\

An important  aspect to consider for  FX interventions is the  source of shock
that the central bank would like  to mitigate. In micro-founded optimal policy
frameworks, the rationale for FX interventions depends on the shock generating
the exchange rate  movement. For example, in \cite{basu2020}, exchange
rate movements  owing to permanent  real shocks―for example,  productivity and
commodity  prices,  and fundamental  changes  in  world interest  rates―should
generally be accommodated unless they trigger financial constraints. Financial
stability depends both  on cyclical (exchange rate  volatility) and structural
factors such that domestic FX hedging,  not just the exchange rate volatility,
could motivate  the central banks  to smooth movements associated  with higher
uncovered interested  rate parity (UIP)  premia. Therefore, according  to this
literature, only  those exchange  rate movements associated  with identifiable
global financial shocks and growing bid-ask  spreads should be included in the
rule,  while movements  arising  from  commodity price  shocks  should not  be
included. However, in  practice, the diagnosis of the  shock requires judgment
(\cite{basu2020}, \cite{cavallino2019}), and  identifying in  real time  the
source of the shock is often not possible. Therefore, the FX intervention rule
we present here has a somewhat narrow focus. The rationale for the VaR rule is
to preserve orderly market conditions by preventing excess volatility and tail
risks to  materialize, irrespective of  their source. One important  aspect is
that  any  source  of  shocks  could  potentially  degenerate  into  financial
stability risks, as long as it creates risks to unhedged exposure. This is the
reason we do  not identify the source  of the shock in this  paper, which also
has benefits in terms of implementation.\\

In  this  paper, an  intervention  is  deemed rule  based  when  it reacts  to
predetermined parameters to deliver predictable responses. The most used rules
are  based  on fixed-volatility  triggers  such  as day-to-day  exchange  rate
change, for  example, 2  percent depreciation from  the previous  day exchange
rate close.   The rule can  be disclosed or kept  secret by the  central bank,
although,   in   the   latter   case,   it   may   become   transparent   with
experience. \cite{patel2019} surveyed 21 emerging market central banks and six
out  the   21  regularly  use   an  intervention   rule,  while  four   do  so
occasionally.\\

Some studied the  efficiency of FX rules and usually  find them less efficient
than discretion. However,  in some cases, the understanding  of rules includes
tactics such as  "leaning against the wind," that is,  delaying the adjustment
of the  exchange rate,  which would  not be  considered a  rule in  this paper
\citep{chutasripanich2015}. In other cases, it  involves central banks with an
already  established  preference  for   floating,  for  which  the  literature
indicates that the  rule may not be superior  to discretion \citep{fatum2005}.
While   country-specific   empirical    work   on   rule-based   interventions
(fixed-volatility  triggers) was  performed for  Canada \citep{fatum2005}  and
Columbia  \citep{kuersteiner2018},  none  was  completed for  Mexico,  to  our
knowledge.\\

The concept of VaR, as formalized in \cite{jorion2007}, is frequently used for
financial applications, for managing  risk exposure, for portfolio allocation,
and so  on. \cite{alexander2009}  provides a comprehensive  review of  VaR for
market risk analysis.  Among many applications  of the VaR model, one can cite
in the  FX field \cite{aljanabi2006}, who  proposes to consistently use  a VaR
framework for managing trading risk exposure  of FX securities, in the context
of emerging and illiquid markets.  \cite{bredin2004} review the performance of
a number of VaR methods using a portfolio  based on the FX exposure of a small
open economy.\\

Using  ARCH/GARCH  (Autoregressive Conditional  Heteroskedasticity/Generalized
Autoregressive Conditional  Heteroskedasticity) models  for estimating  VaR is
also  standard  practice  in  the  literature.   \cite{engle2001}  conducts  a
comprehensive overview of the ARCH/GARCH  models in financial econometrics and
devotes an entire section to  estimating VaR.  \cite{giot2004} model daily VaR
using realized  volatility and  ARCH models,  and show  that it  has excellent
forecasting  performances.   \cite{chan2007}  use nonlinear  GARCH  models  to
estimate  VaR  in  the  presence  of a  data  generating  process  with  heavy
tails. Other types of  models could be used to estimate  VaR, such as quantile
regressions    \citep{gaglianone2011},    copulas   \citep{patton2001},    and
nonparametric  kernel \citep{hoogerheide2010}.   However, the  GARCH model  is
used for this paper because it is a standard model used by market participants
and central  banks around  the world, with  widespread implementation  on many
statistical packages.  Besides, as \cite{jeon2002}  show, FX markets are quite
efficient  and their  features  fit well  the simple  and  robust approach  of
standard GARCH models.\\

Finally,  GARCH  models  are  frequently  used for  the  analysis  of  the  FX
markets. \cite{hansen2005}  argue that in  the context of daily  exchange rate
returns, nothing can  beat a GARCH(1,1) model,  while \cite{mcmillan2012} show
that  an  intraday  GARCH(1,1)  model generally  provides  superior  forecasts
compared with all other models.\\

%% ---------------------------------------------------------------------------
%% VaR Interventions and Exchange Rate Value-at-Risk
%% ---------------------------------------------------------------------------
\section{VaR Interventions and Exchange Rate Value-at-Risk}
\label{sec:var-interventions}

%% ---------------------------------------------------------------------------
%% Bibliography
%% ---------------------------------------------------------------------------
\newpage
\singlespacing
\bibliographystyle{apalike2}
\bibliography{bibliography}

%% ---------------------------------------------------------------------------
%% Annex
%% ---------------------------------------------------------------------------
\newpage
\appendix 
\section{Annex}

%% ---------------------------------------------------------------------------
%% End the document
%% ---------------------------------------------------------------------------
\end{document}
