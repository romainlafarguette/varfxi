% Time-stamp: <2020-09-12 23:00:26 Romain>
% Beamer template for the Macro Stress-Testing Article
% August 2020
% Contact: Romain Lafarguette, rlafarguette@imf.org

%% ---------------------------------------------------------------------------
%% Preamble: Packages and Setup
%% ---------------------------------------------------------------------------
% Class 
\documentclass{beamer}

% Language
\usepackage[english]{babel}

% Font and encoding
\usepackage[utf8]{inputenc} % Input font
\usepackage[T1]{fontenc} % Output font
\usepackage{lmodern} % Standard LateX font
\usefonttheme{serif} % Standard LateX font

% Maths 
\usepackage{amsfonts, amsmath, mathabx, bm, bbm} % Maths Fonts

% Graphics
\usepackage{graphicx} % Insert graphics
\usepackage{subfig} % Multiple figures in one graphic
\graphicspath{{../output/}{../img/}}

% Colors
\usepackage{xcolor}
\definecolor{imfblue}{RGB}{0,76,151} % Official IMF color
\setbeamercolor{title}{fg=imfblue}
\setbeamercolor{frametitle}{fg=imfblue}
\setbeamercolor{structure}{fg=imfblue}
\setbeamercolor{page number in head/foot}{fg=imfblue}
\setbeamerfont{page number in head/foot}{size=\footnotesize}

% Tables
\usepackage{booktabs,rotating,multirow} % Tabular rules and other macros
\usepackage{pdflscape,afterpage} % Landscape mode and afterpage
\usepackage{threeparttable} % Split long tables
\usepackage[font=scriptsize,labelfont=scriptsize,labelfont={color=imfblue}]{caption}

% Import files
\usepackage{import}

% Appendix slides
\usepackage{appendixnumberbeamer} % Manage page numbers for appendix slides

% Bibliographies
\usepackage{natbib} % Author-year bibliography style

% A few macros: environments
\newenvironment{largeitemize}{\itemize\addtolength{\itemsep}{10pt}}{\enditemize}
\newenvironment{largeenumerate}{\enumerate\addtolength{\itemsep}{10pt}}{\endenumerate}
\newenvironment{wideitemize}{\itemize\addtolength{\itemsep}{30pt}}{\enditemize}
\newenvironment{wideenumerate}{\enumerate\addtolength{\itemsep}{30pt}}{\endenumerate}

% Define the footer
\makeatother
\setbeamertemplate{footline}[page number]
\newcommand{\myframe}[1]{%
    \begin{frame}
        \begin{center}
            \Large #1 slide \insertframenumber/\inserttotalframenumber\\
            Page number \thepage/\inserttotalframenumber
        \end{center}
    \end{frame}
}
\makeatletter

% Remove navigation symbols and other superfluous elements
\setbeamertemplate{navigation symbols}{}
\setbeameroption{hide notes}
\setbeamertemplate{note page}[plain]
\beamertemplatenavigationsymbolsempty
\hypersetup{pdfpagemode=UseNone} % don't show bookmarks on initial view

% Institute font
\setbeamerfont{institute}{size=\footnotesize}
\DeclareMathSizes{10}{9}{7}{5}  

%% ---------------------------------------------------------------------------
%% Title info
%% ---------------------------------------------------------------------------
\title[]{FX Interventions Rules for Central Banks\\
A Risk-Based Framework}
\author[]{Romain Lafarguette \and Romain Veyrune}
\institute[]{MCM Central Bank Operations Division}

\date[]{\scriptsize September 16, 2020 \\ \vspace{0.5cm} \scriptsize{\textit{The views
      expressed in this presentation do not necessarily represent the views of
      the IMF, its Executive Board, or IMF management.}} \vspace{-0.3cm}}

\titlegraphic{
    \begin{figure}
    \centering
    \subfloat{{\includegraphics[width=2cm]{imf_logo}}}%
    \end{figure}
}


% Insert the plan at each beginning of the section
\AtBeginSection[]
  {
     \begin{frame}
     \frametitle{Table of Contents}
     \tableofcontents[currentsection, hideothersubsections]
     \end{frame}
  }

%% ---------------------------------------------------------------------------
%% Title slide
%% ---------------------------------------------------------------------------
\begin{document}

\begingroup
\renewcommand{\insertframenumber}{}
\begin{frame}
  \addtocounter{framenumber}{-1}
\maketitle
\end{frame}
\endgroup


%% ---------------------------------------------------------------------------
%% Key Messages
%% ---------------------------------------------------------------------------
\section{Key messages}
\begin{frame}{Key Messages}

  Foreign Exchange intervention rules should\\
  \medskip
  
  \begin{largeitemize}
  \item Depend on market conditions
  \item Objective, be anchored to a risk tolerance level
    rather than an aribtrary FX level threshold
  \item Capture FX non-linearities and asymmetries between appreciation and
    depreciation
  \item Be forward-looking
  \item Be easily operationalizable\\
  \end{largeitemize}

\medskip  
We propose an FX intervention rule based on \textbf{Conditional Value at Risk}
  
\end{frame}


%% ---------------------------------------------------------------------------
%% Model
%% ---------------------------------------------------------------------------
\section{Model}
\begin{frame}{Regression Table}
\setlength\tabcolsep{2pt}  % default value: 6pt
\tiny  %%  command to change the font size
\begin{tabular}{lllllll}
\toprule
{} & Constant & Microstructure &       CIP &       FXI &  Baseline & Robustness \\
\midrule
Intercept                       &     1.09 &          -2.16 &      2.15 &   1.67*** &      1.63 &    1.64*** \\
Lag FX log returns              &  0.09*** &        0.08*** &   0.08*** &   0.08*** &   0.08*** &    0.08*** \\
Bid-ask spread abs value        &          &         0.11** &   0.15*** &   0.14*** &   0.15*** &    0.15*** \\
Forward points first difference &          &        0.32*** &   0.32*** &   0.32*** &   0.27*** &    0.27*** \\
Interbank rate vs Libor         &          &                &  -1.11*** &  -0.98*** &  -1.02*** &   -1.03*** \\
FX intervention in USD lag      &          &                &           &      0.04 &      0.04 &            \\
VIX first diff                  &          &                &           &           &   9.78*** &    9.79*** \\
EURUSD log returns              &          &                &           &           &   0.13*** &    0.13*** \\
FX intervention dummy lag       &          &                &           &           &           &       4.13 \\
Omega                           &  0.15*** &        0.14*** &   0.13*** &   0.13*** &   0.14*** &    0.14*** \\
Alpha                           &  0.17*** &        0.19*** &   0.18*** &   0.18*** &   0.19*** &    0.19*** \\
Gamma                           &  0.06*** &        0.06*** &   0.06*** &   0.05*** &   0.05*** &    0.05*** \\
Beta                            &  0.98*** &        0.98*** &   0.98*** &   0.99*** &   0.98*** &    0.98*** \\
Nu                              &  8.81*** &        9.11*** &   9.18*** &   9.15*** &   7.77*** &    7.77*** \\
Lambda                          &  0.13*** &        0.11*** &   0.12*** &   0.12*** &    0.1*** &     0.1*** \\
R2                              &    0.4 \% &          4.9 \% &     5.1 \% &     5.1 \% &    14.3 \% &     14.3 \% \\
R2 adjusted                     &    0.4 \% &          4.8 \% &     5.0 \% &     5.0 \% &    14.2 \% &     14.1 \% \\
Number of observations          &     4511 &           4511 &      4511 &      4510 &      4510 &       4510 \\
Significance *10\%, **5\%, ***1\%  &          &                &           &           &           &            \\
\bottomrule
\end{tabular}

\normalsize
\end{frame}


%% ---------------------------------------------------------------------------
%% In-sample dynamics
%% ---------------------------------------------------------------------------
\section{In-sample dynamics}
\begin{frame}[plain]
\frametitle{Dynamics of the Mexican Peso against USD}
    \makebox[\linewidth]{\includegraphics[width=\paperwidth]{descriptive_plot.pdf}}
\end{frame}


\begin{frame}[plain]
\frametitle{Conditional In-Sample Volatility of the Mexican Peso}
    \makebox[\linewidth]{\includegraphics[width=\paperwidth]{conditional_vol_plot.pdf}}
\end{frame}


%% ---------------------------------------------------------------------------
%% Forecasting
%% ---------------------------------------------------------------------------
\begin{frame}[plain]
\frametitle{Conditional Distribution of the Mexican Peso over Time}
    \makebox[\linewidth]{\includegraphics[width=\paperwidth]{joyplot.pdf}}
\end{frame}






  

%% ---------------------------------------------------------------------------
%% End document
%% ---------------------------------------------------------------------------
\end{document}

