% Time-stamp: <2020-09-14 01:28:17 Romain>
% Beamer template for the Macro Stress-Testing Article
% August 2020
% Contact: Romain Lafarguette, rlafarguette@imf.org

%% ---------------------------------------------------------------------------
%% Preamble: Packages and Setup
%% ---------------------------------------------------------------------------
% Class 
\documentclass{beamer}

% Language
\usepackage[english]{babel}

% Font and encoding
\usepackage[utf8]{inputenc} % Input font
\usepackage[T1]{fontenc} % Output font
\usepackage{lmodern} % Standard LateX font
\usefonttheme{serif} % Standard LateX font

% Maths 
\usepackage{amsfonts, amsmath, mathabx, bm, bbm} % Maths Fonts

% Graphics
\usepackage{graphicx} % Insert graphics
\usepackage{subfig} % Multiple figures in one graphic
\graphicspath{{../output/}{../img/}}

% Colors
\usepackage{xcolor}
\definecolor{imfblue}{RGB}{0,76,151} % Official IMF color
\setbeamercolor{title}{fg=imfblue}
\setbeamercolor{frametitle}{fg=imfblue}
\setbeamercolor{structure}{fg=imfblue}
\setbeamercolor{page number in head/foot}{fg=imfblue}
\setbeamerfont{page number in head/foot}{size=\footnotesize}

% Tables
\usepackage{booktabs,rotating,multirow} % Tabular rules and other macros
\usepackage{pdflscape,afterpage} % Landscape mode and afterpage
\usepackage{threeparttable} % Split long tables
\usepackage[font=scriptsize,labelfont=scriptsize,labelfont={color=imfblue}]{caption}

% Import files
\usepackage{import}

% Appendix slides
\usepackage{appendixnumberbeamer} % Manage page numbers for appendix slides

% Bibliographies
\usepackage{natbib} % Author-year bibliography style

% A few macros: environments
\newenvironment{largeitemize}{\itemize\addtolength{\itemsep}{10pt}}{\enditemize}
\newenvironment{largeenumerate}{\enumerate\addtolength{\itemsep}{10pt}}{\endenumerate}
\newenvironment{wideitemize}{\itemize\addtolength{\itemsep}{30pt}}{\enditemize}
\newenvironment{wideenumerate}{\enumerate\addtolength{\itemsep}{30pt}}{\endenumerate}

% Define the footer with higher/lower adjustment
\defbeamertemplate{footline}{higher page number}
{
  \hfill
  \usebeamercolor[fg]{page number in head/foot}
  \usebeamerfont{page number in head/foot}  
  \thepage/\inserttotalframenumber\kern1em\vskip2pt %Change xxpt to
                                %lower/higher the footnote
}
\setbeamertemplate{footline}[higher page number]


% Remove navigation symbols and other superfluous elements
\setbeamertemplate{navigation symbols}{}
\setbeameroption{hide notes}
\setbeamertemplate{note page}[plain]
\beamertemplatenavigationsymbolsempty
\hypersetup{pdfpagemode=UseNone} % don't show bookmarks on initial view

% Institute font
\setbeamerfont{institute}{size=\footnotesize}
\DeclareMathSizes{10}{9}{7}{5}  

%% ---------------------------------------------------------------------------
%% Title info
%% ---------------------------------------------------------------------------
\title[]{FX Interventions Rules for Central Banks\\
A Risk-Based Framework}
\author[]{Romain Lafarguette \and Romain Veyrune}
\institute[]{IMF Monetary and Capital Markets Department \\ Central Bank Operations Division}

\date[]{\scriptsize September 16, 2020 \\ \vspace{0.5cm} \scriptsize{\textit{The views
      expressed in this presentation do not necessarily represent the views of
      the IMF, its Executive Board, or IMF management.}} \vspace{-0.3cm}}

\titlegraphic{
    \begin{figure}
    \centering
    \subfloat{{\includegraphics[width=2cm]{imf_logo}}}%
    \end{figure}
}


% Insert the plan at each beginning of the section
\AtBeginSection[]
  {
     \begin{frame}
     \frametitle{Table of Contents}
     \tableofcontents[currentsection, hideothersubsections]
     \end{frame}
  }

%% ---------------------------------------------------------------------------
%% Title slide
%% ---------------------------------------------------------------------------
\begin{document}

\begingroup
\renewcommand{\insertframenumber}{}
\begin{frame}
  %\addtocounter{framenumber}{-1}
\maketitle
\end{frame}
\endgroup


%% ---------------------------------------------------------------------------
%% Framework
%% ---------------------------------------------------------------------------
\section{Framework}

\begin{frame}{Contributions}

  \begin{largeitemize}
    \item Design a rule for central banks that intervene to prevent \textbf{disorderly
        market conditions}
    \item Provides guidance on \textbf{when} to intervene ("triggers")
    \item Appropriate for \textbf{floating exchange rate regimes} with FX
      risks to the economy (e.g. FX unhedged exposures)
    \item Consistenty control \textbf{FX risk} rather than arbitrary FX volatility/level threshold
    \item A \textbf{risk management framework} for central banks' financial
      stability mandate: aligned with \textbf{industry's best practices} in FX management
  \end{largeitemize}
  
\end{frame}


\begin{frame}
  \frametitle{\color{red}{What the rule is NOT about}}
  \begin{alertblock}{}
    \begin{largeitemize}
    \item Not designed to reach or to preserve a given FX level
      (e.g. alignment with the \textbf{equilibrium exchange rate level})
    \item Doesn't prevent \textbf{appreciation/depreciation trends} to occur...
    \item ... but can be compatible with other approaches, e.g. discretionary FXI
    \item We \textbf{don't discuss the efficiency} of FX interventions from a
      welfare/macro point of view
    \item Not a guide to calibrate \textbf{FX interventions amount}
    \item Not a guide for the optimal \textbf{currency allocation} of FX reserves
    \end{largeitemize}    
  \end{alertblock}
  
\end{frame}



\begin{frame}{Key Messages}
  Foreign Exchange intervention rules should\\
  \medskip
  
  \begin{largeitemize}
  \item Depend on market conditions
  \item Objective, be anchored to a risk tolerance level
    rather than an aribtrary FX level threshold
  \item Capture FX non-linearities and asymmetries between appreciation and
    depreciation
  \item Be easily operationalizable\\
  \end{largeitemize}

\medskip  
We propose an FX intervention rule based on \textbf{Conditional Value-at-Risk}  
\end{frame}


\begin{frame}{Concept: Value-at-Risk Rule}

  \begin{largeitemize}
    \item Rather than using a fixed volatility rule (e.g. intervene if daily
      exchange rate varies by more than 2\%)
    \item Use a \textbf{risk-based rule}: intervene when the daily exchange rate log-returns are in the
      tails of the conditional distribution
    \item Measure the tail-risk via the concept of \textbf{Value-at-Risk} (the
      conditional quantile of the log returns distribution) 
    \item The conditional distribution is estimated daily with a standard
      financial GARCH model and varies with market conditions
    \item The central bank decides on the \textbf{risk tolerance}:
      e.g. intervene in the tail at 1\%, 5\%, 10\%, etc.
  \end{largeitemize}
\end{frame}


\begin{frame}
  \frametitle{VaR FXI Rule}
    \makebox[\linewidth]{\includegraphics[width=\paperwidth]{var_rule.pdf}}
\end{frame}


\begin{frame}{A Risk-Management Approach to FX Interventions}
  \begin{largeitemize}
    \item Under this framework, central bank interventions prevent FX tail
      risks to occur
    \item This rule allows flexible exchange rate to act as a shock absorber:
      don't prevent FX variation 
    \item The central bank risk tolerance should be fixed in accordance with the
      macrofinancial risk in the economy (unhedged exposures from resident agents, degree of
      dollarization, etc.) as well as market resilience
    \item The financial stability mandate of the central bank is properly
      formalized and quantified. Aligned with industries' practice in risk management.
  \end{largeitemize}
  
\end{frame}


\begin{frame}{Features}
  \begin{largeenumerate}
    \item Prevent moral hazard and market manipulation, and align the financi
    
    \item This rule guarantees that the interventions will occur with a \textbf{fixed
        frequency} over the medium term
    \item 
  \end{largeenumerate}
  
\end{frame}





%% ---------------------------------------------------------------------------
%% Model
%% ---------------------------------------------------------------------------
\section{Model}

\begin{frame}{Specification}
  
Non-linear, Exponential GARCH (EGARCH) model to estimate the conditional
VaR of exchange rate log-retrurns $r_t$, $Q(r_t| \Omega_{t-1}, \theta)$

\bigskip

\begin{largeitemize}
\item[] \textbf{Drift AR-X(1):} $r_{t+1} = \text{Intercept} +
  \rho r_t + \beta X_{t+1} + \epsilon_{t-1}$\\
  
\item[] \textbf{Exponential volatility:} $\log \sigma_{t+1}^{2} = \omega + \beta
g(r_t)$ where $g(r_t) = \alpha r_t + \gamma(|r_t|-\mathbb{E}|r_t|)$

\item[] \textbf{Error term distribution} $\epsilon_t = \sigma_t \varepsilon_t,
  \ , \varepsilon_t \sim \text{TSK}(k,v)$\\
\end{largeitemize}

\bigskip

\begin{largeitemize}
  \item The GARCH estimation is standard and done with MLE; selection of parameters
    is done via AIC/BIC criteria.
  \item The Python package we designed allows to select the control variables, choose
the lag, fit different distributions, etc.
\end{largeitemize}
\end{frame}

\begin{frame}{Exogeneous Factors}

  \begin{largeenumerate}
  \item FX bid-ask spread (averaged over the day) => \textbf{FX microstructure}
  \item Daily interest rate differential with the US Libor => \textbf{CIP}
  \item The one-month forward exchange rate => \textbf{cost of hedging} on the spot
    market
  \item The lagged amount of central bank FX intervention => \textbf{policy interventions}
  \item The VIX => \textbf{global risk sentiment}
  \item The EURUSD exchange rate => control for \textbf{global FX factor}
  \end{largeenumerate}
 
\end{frame}


\begin{frame}{Regression Table}
\setlength\tabcolsep{2pt}  % default value: 6pt
\tiny  %%  command to change the font size
\begin{tabular}{lllllll}
\toprule
{} & Constant & Microstructure &       CIP &       FXI &  Baseline & Robustness \\
\midrule
Intercept                       &     1.09 &          -2.16 &      2.15 &   1.67*** &      1.63 &    1.64*** \\
Lag FX log returns              &  0.09*** &        0.08*** &   0.08*** &   0.08*** &   0.08*** &    0.08*** \\
Bid-ask spread abs value        &          &         0.11** &   0.15*** &   0.14*** &   0.15*** &    0.15*** \\
Forward points first difference &          &        0.32*** &   0.32*** &   0.32*** &   0.27*** &    0.27*** \\
Interbank rate vs Libor         &          &                &  -1.11*** &  -0.98*** &  -1.02*** &   -1.03*** \\
FX intervention in USD lag      &          &                &           &      0.04 &      0.04 &            \\
VIX first diff                  &          &                &           &           &   9.78*** &    9.79*** \\
EURUSD log returns              &          &                &           &           &   0.13*** &    0.13*** \\
FX intervention dummy lag       &          &                &           &           &           &       4.13 \\
Omega                           &  0.15*** &        0.14*** &   0.13*** &   0.13*** &   0.14*** &    0.14*** \\
Alpha                           &  0.17*** &        0.19*** &   0.18*** &   0.18*** &   0.19*** &    0.19*** \\
Gamma                           &  0.06*** &        0.06*** &   0.06*** &   0.05*** &   0.05*** &    0.05*** \\
Beta                            &  0.98*** &        0.98*** &   0.98*** &   0.99*** &   0.98*** &    0.98*** \\
Nu                              &  8.81*** &        9.11*** &   9.18*** &   9.15*** &   7.77*** &    7.77*** \\
Lambda                          &  0.13*** &        0.11*** &   0.12*** &   0.12*** &    0.1*** &     0.1*** \\
R2                              &    0.4 \% &          4.9 \% &     5.1 \% &     5.1 \% &    14.3 \% &     14.3 \% \\
R2 adjusted                     &    0.4 \% &          4.8 \% &     5.0 \% &     5.0 \% &    14.2 \% &     14.1 \% \\
Number of observations          &     4511 &           4511 &      4511 &      4510 &      4510 &       4510 \\
Significance *10\%, **5\%, ***1\%  &          &                &           &           &           &            \\
\bottomrule
\end{tabular}

\normalsize
\end{frame}




%% ---------------------------------------------------------------------------
%% In-sample dynamics
%% ---------------------------------------------------------------------------
\section{In-sample dynamics}
\begin{frame}
\frametitle{Dynamics of the Mexican Peso against USD}
    \makebox[\linewidth]{\includegraphics[width=\paperwidth]{descriptive_plot.pdf}}
\end{frame}


\begin{frame}
\frametitle{Conditional In-Sample Volatility of the Mexican Peso}
    \makebox[\linewidth]{\includegraphics[width=\paperwidth]{conditional_vol_plot.pdf}}
\end{frame}

%% ---------------------------------------------------------------------------
%% Forecasting
%% ---------------------------------------------------------------------------
\section{Forecasting}
\begin{frame}
 %\frametitle{Conditional Distributions}
\makebox[\linewidth]{\includegraphics[width=\paperwidth, height=\paperheight]{joyplot.pdf}}
\end{frame}

\begin{frame}
 \frametitle{Fan Chart}
\makebox[\linewidth]{\includegraphics[width=0.95\paperwidth]{fanchart.pdf}}
\end{frame}

\begin{frame}
  \frametitle{VaR FXI Rule}
    \makebox[\linewidth]{\includegraphics[width=\paperwidth]{var_rule.pdf}}
\end{frame}

\begin{frame}
  \frametitle{Conditional Cumulative Distribution Function}
    \makebox[\linewidth]{\includegraphics[width=\paperwidth]{conditional_cdf.pdf}}
\end{frame}

\begin{frame}
  \frametitle{Conditional Exceedance}
    \makebox[\linewidth]{\includegraphics[width=\paperwidth]{conditional_exceedance.pdf}}
\end{frame}


\begin{frame}
  \frametitle{Density Evaluation}
    \makebox[\linewidth]{\includegraphics[width=\paperwidth]{pitchart.pdf}}
\end{frame}


%% ---------------------------------------------------------------------------
%% Benchmarking
%% ---------------------------------------------------------------------------
\section{Benchmarking}


\begin{frame}
  \frametitle{Rule-Based Benchmarking}
    \makebox[\linewidth]{\includegraphics[width=\paperwidth]{benchmark_minprice.pdf}}
\end{frame}


\begin{frame}
  \frametitle{Rule-Based Benchmarking: Risk-Level}
    \makebox[\linewidth]{\includegraphics[width=\paperwidth]{benchmark_minprice_cdf.pdf}}
\end{frame}


\begin{frame}
  \frametitle{Discretion-Based Benchmarking}
    \makebox[\linewidth]{\includegraphics[width=\paperwidth]{benchmark_no_minprice.pdf}}
\end{frame}


\begin{frame}
  \frametitle{Discretion-Based Benchmarking: Risk-Level}
    \makebox[\linewidth]{\includegraphics[width=\paperwidth]{benchmark_no_minprice_cdf.pdf}}
\end{frame}




  

%% ---------------------------------------------------------------------------
%% End document
%% ---------------------------------------------------------------------------
\end{document}

